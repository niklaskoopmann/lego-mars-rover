\chapter{Fazit}
\label{chp:fazit}

Im Rahmen dieser Studienarbeit ist ein weitgehend funktionsfähiges und nutzbares Mars-Rover-Modell nach dem realen Vorbild, dem Curiosity-Rover der \acs{nasa}, in Form einer Machbarkeitsstudie (Proof of Concept) von Grund auf entwickelt worden.
Das Ergebnis entspricht gänzlich den in Kapitel \ref{chp:spezifikation} definierten und im Verlauf der Entwicklung angepassten Anforderungen.
Besonderes Augenmerk wurde bereits bei der Konzeption des Rovers auf die originalgetreue und formschöne Abbildung wichtiger Komponenten wie dem Rocker-Bogie-Fahrwerk mit sechsrädrigem Antriebssystem und der Kamerapositionierung im Kopf gelegt.
Zudem sind charakteristische Strukturelemente von Curiosity, zum Beispiel der stets leicht geneigte Kopf oder der \enquote{Bienenstock} am Heck, in das Design eingeflossen.
Dies ist in Kapitel \ref{chp:modell} ausführlich in Form von Fotografien des realen und Render-Exporten des digitalen Modells aus verschiedenen Perspektiven dokumentiert.
Die Verarbeitung der Eingangsdaten und die Steuerung des Antriebs geschehen zentral auf dem Raspberry Pi unter Nutzung der drei BrickPi-Platinen.
Hierfür sind verschiedene Python-Module von Grund auf entwickelt worden.
Diese Entwicklung sowie die Zusammenarbeit von Software und Hardware sind gemeinsam in Kapitel \ref{chp:implementierung} erläutert worden.
In den Skripts kommen verschiedene Bibliotheken zum Einsatz; sie sind jeweils in den entsprechenden Abschnitten dieses Dokumentes benannt und inklusive der genutzten Versionen in einer separaten Readme-Datei dokumentiert.
Abschließend ist die gesamte Funktionsfähigkeit des Rovers nachgewiesen worden.
In Kapitel \ref{chp:nachweis_leistungsfaehigkeit} ist festgehalten, wie diese Nachweise erbracht wurden.

Im folgenden Abschnitt sollen abschließend noch einige Möglichkeiten der Weiterentwicklung des Modells in verschiedenen Bereichen aufgezeigt werden.

\section{Ausblick}
\label{sec:ausblick}

In diesem Proof of Concept ist bestätigt worden, dass sich ein realitätsnahes Abbild des Mars Rovers Curiosity der \acs{nasa} aus LEGO erstellen und mithilfe von EV3-Motoren betreiben lässt.
Das Ergebnis bildet eine solide Grundlage für weitere Entwicklungen, beispielsweise bezüglich Schwarmintelligenz, künstlicher Intelligenz und Objekterkennung.
Nichtsdestotrotz bietet es einigen Punkten Potential für Verbesserungen.

Insbesondere bei der Kurven- und Geländefahrt ist die Stabilität des LEGO-Modells nicht uneingeschränkt sichergestellt.
Hierbei zeigte sich während der Entwicklung unter bestimmten Bedingungen ein Auseinanderdriften der jeweiligen Räderpaare (wie in Kapitel \ref{chp:implementierung} angesprochen) und seltener ein seitliches Kippen einzelner Räder.
Es ist zu vermuten, dass dies in weiten Teilen auf die ursprüngliche Unterdimensionierung des Fahrwerks (Entwurf für einen deutlich leichteren Korpus) zurückzuführen ist.
Werden die entsprechenden Schwachstellen an den zwei Rotationspunkten jeder Seite des Rocker-Bogie-Fahrwerks sowie den Verbindungen zwischen Lenkmotoren und angetriebenen Rädern verstärkt, ist eine deutliche Steigerung der seitlichen Stabilität des Fahrwerks zu erwarten.
Durch geschickt angebrachte Federung, die Nutzung von Abstandssensoren zur Überwachung des Radstandes oder das Ersetzen einzelner Fahrwerksteile durch stabilere Streben, die sich mithilfe des 3D-Drucks herstellen lassen, kann die Stabilität weiter erhöht werden.

Unter Umständen lässt sich durch die Reduzierung der mitgeführten Powerbanks auf eine einzelne zusätzlich das Gewicht des Korpus deutlich verringern.
Zuvor sollte jedoch die Nutzbarkeit eines entsprechenden 1-zu-3-Splitters auch unter länger andauernder Systemlast geprüft werden.
Die produktive Nutzung des angebrachten Solarmoduls (siehe Abbildung \ref{fig:rovertopfoto}) könnte in diesem Fall bei der Nutzung unter freiem Himmel zu einer messbaren Reduzierung der Ladestopps führen.
Für eine längere voll-autonome Nutzung ist die Selbstversorgung mit Energie ohnehin unabdingbar.

Bezüglich der Weiterentwicklung im Studiengang Angewandte Informatik sollte der Fokus allerdings auf einen Ausbau der Softwarefunktionalität gelegt werden.
Insbesondere im Bereich der Objekterkennung bietet das Modell viel Potential.
So kann beispielsweise in nächsten Schritten eine Klassifikation auf Basis künstlicher Intelligenz erfolgen.
Der Rover wäre durch diese Technologie in der Lage, neben blauen LEGO-Steinen zum Beispiel auch echte Wasserpfützen und Hindernisse wie Steine und Löcher zu erkennen und zu unterscheiden.

Auch im Bereich der Schwarmintelligenz bieten sich Möglichkeiten der Weiterentwicklung.
So lässt sich die Erkundung einer Marslandschaft deutlich verkürzen, wenn diese zeitgleich durch mehrere Rover geschieht, die untereinander kommunizieren und ihre Betrachtungsgebiete aufteilen.