\chapter{Fazit}
\label{chp:fazit}

Im Rahmen dieser Studienarbeit ist ein weitgehend funktionsfähiges und nutzbares Mars-Rover-Modell nach dem realen Vorbild, dem \textit{Curiosity}-Rover der \acs{nasa}, in Form einer Machbarkeitsstudie (Proof of Concept) von Grund auf entwickelt worden.
Das Ergebnis entspricht gänzlich den in Kapitel \ref{chp:spezifikation} definierten und im Verlauf der Entwicklung angepassten Anforderungen.
Besonderes Augenmerk wurde bereits bei der Konzeption des Rovers auf die originalgetreue Abbildung wichtiger Komponenten wie dem Rocker-Bogie-Fahrwerk und der Kameraposition gelegt.
Zudem sind charakteristische Strukturelemente von \textit{Curiosity}, zum Beispiel der stets leicht geneigte Kopf oder der \enquote{Bienenstock} am Heck, in das Design eingeflossen.


\section{Ausblick}
\label{sec:ausblick}

- Entwicklung Greifarm entsprechend der Spezifikation

- Stabilität bei Kurven- und Geländefahrt erhöhen

- Reduktion auf eine Powerbank (Option Splitter?)

- mglw. Solarmodul nutzen und Powerbank damit speisen

- echte, KI-basierte Objekterkennung (z. B. Pfützen, Steine und Löcher) -> besser auf RPi4 upgraden -> Kompatibilität zu BrickPi3 checken