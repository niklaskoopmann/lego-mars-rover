\chapter{Implementierung}
\label{chp:implementierung}

\section{Installation und Konfiguration Raspberry Pi}
\label{sec:inst_konf_raspi}

- Raspberry Pi 3 Model B Plus Rev 1.3

- Dexter Industries Raspbian for Robots -> Raspbian-basierte Distribution mit zusätzlichen vorinstallierten Anwendungen und Treibern -> unter anderem für den BrickPi von Dexter Industries

\section{Installation und Konfiguration Kamera}
\label{sec:inst_konf_kamera}

Für die Videoeingabe wird die originale Raspberry Pi Camera V2 genutzt.

Die Objekterkennung für Wasserobjekte erfolgt mithilfe der Programmbibliotheken picamera und OpenCV.
Erstere stellt eine Schnittstelle zwischen der Raspberry-Pi-Kamera und Python zur Verfügung.
Letztere ermöglicht die Echtzeitverarbeitung der Kamerabilder:
Die Objekte werden auf Basis ihrer blauen Farbe erkannt, da sich diese deutlich von der charakteristischen Farbe der Vorbildlandschaften des Mars unterscheidet.
Letztere sei eine \enquote{predominantly yellowish brown color with only subtle variation} \cite{maki1999}.
Die LEGO-Steine, welche als Wasserobjekte fungieren, haben die LEGO-Farb-ID $7$.
Diese korrespondiert mit dem Farbcode \#0057A6 in Hexadezimaldarstellung.