\chapter{Nachweis der Leistungsfähigkeit}
\label{chp:nachweis_leistungsfaehigkeit}

Der Nachweis der Leistungsfähigkeit entsprechend der angepassten Anforderungen wird durch die folgenden Dokumentationsteile erbracht:

Das BrickLink-Modell wird möglichst originalgetreu gehalten und dokumentiert die Struktur des Rovers.
Durch verschiedene Teile, die unter bestimmten Bedingungen (aufgebockt, stehend, fahrend et cetera) unter Spannung stehen und bewegliche Teile kann das digitale Modell nicht exakt die reale LEGO-Struktur abbilden.
An einzelnen Stellen sind derartige Teile (häufig beschränkt auf einzelne Pins) ausgelassen worden, um Überschneidungen im digitalen Modell zu vermeiden.
Die Teile sind aber in den verschiedenen Fotografien des Rovers (siehe Kapitel \ref{chp:modell}) zu erkennen.

Durch Videoaufnahmen verschiedener Einsatzszenarien ist die Leistungsfähigkeit der einzelnen Anforderungen nachgewiesen.
In einer ersten Aufnahme klassifiziert der Rover stehend Wasserobjekte, die vor ihm auf dem Boden verteilt werden, und es erfolgt eine Sprachausgabe des spezifizierten Textes \enquote{Water found}.
Eine weitere Aufnahme zeigt den Rover stehend, während er wiederholt Wasserobjekte in einer Hand erkennt, die abwechselnd verdeckt und präsentiert werden.
Hierbei wird besonders die Sprachausgabe hörbar, da der Rover aus der Nähe gefilmt wird.
Die Fahrtauglichkeit wird durch eine Aufnahme dokumentiert, in welcher dem Rover ein Sprachkommando \enquote{Start} gegeben wird, woraufhin er aus dem Stand losfährt, bis er nach drei Sekunden wieder zum Stehen kommt.
Für die Spracheingabe muss zum aktuellen Zeitpunkt ein kabelgebundenes Mikrofon genutzt werden, da die Bestellung eines neuen Mikrofons aufgrund äußerer Einflüsse nicht möglich war.
Zuletzt wird die Rotationsfahrt des Rovers in Form zweier Videoaufnahmen dokumentiert.
Die Sprachbefehle \enquote{Move left} und \enquote{Move right} initiieren respektive die Links- und Rechtsrotation, die entsprechend der Beschreibung in Kapitel \ref{chp:implementierung} ausgeführt wird.
Am Ende seiner Rotation entdeckt der Rover vor ihm liegende Wasserobjekte um kommuniziert die Erkennung per Sprachausgabe.
Es wird jeweils einmal die Links- und die Rechtsrotation nach diesem Vorgehen dokumentiert.

Neben dem digitalen Modell und den Videoaufnahmen dient auch der Inhalt dieser Arbeit der ausführlichen Dokumentation des entwickelten Rovers.
Die Python-Quelltexte der Steuerungsskripts werden ebenfalls zur Bewertung eingereicht und im Nachgang auf GitHub veröffentlicht.
Dabei ist zu beachten, dass bei Änderungen an der Standardkonfiguration die einwandfreie Funktionsfähigkeit der einzelnen Komponenten nicht gewährleistet werden kann.
Auch eine ausführliche Bauanleitung für das LEGO-Modell wird mithilfe von BrickLink Studio 2.0.10 erstellt und als Teil der Dokumentation eingereicht.