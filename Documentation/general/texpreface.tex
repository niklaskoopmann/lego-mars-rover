\documentclass[ %
%draft,
12pt, 				% % % Schriftgröße 12 Punkte
a4paper,			% % % Papierformat DIN A4 210x297mm
oneside,			% % % Einseitiger Druck
openany, 			% % % Kann auf der linken oder rechten Seite beginnen (bei onside nicht notwendig)
titlepage,			% % % Der Titel kommt auf eine separate Seite
openbib,			% % % Verwendet openbib für die Literatur
listof=totoc,		% % % Zeigt Verzeichnisse Tabellenverzeichnis und Abbildungsverzeichnis im Inhalt an
bibliography=totoc,	% % % Zeigt das Literaturverzeichnis im Inhaltsverzeichnis an
numbers=noenddot,	% % % Lässt den abschließenden Punkt bei Abschnittsnummern weg
headsepline,		% % % Trennlinie zwischen Kopfzeile und Inhalt
headings=onelinechapter, % Kompakte Kapitelüberschriften
]{scrbook}			% % %  KOMA-Script für den Dokumententyp Buch


% % % Für eine saubere Encodierung und die deutsche Sprache
\usepackage[utf8]{inputenc}				% Lese UTF-8 Zeichen als Eingabe
\usepackage[T1]{fontenc}				% Schriftkodierung
\usepackage[ngerman]{babel}				% Deutsche Lokalisierung -> Standardbegriffe in Deutsch anstelle von Englisch
\usepackage{textcomp}					% Zusätzliche Symbole
%\usepackage[charter]{mathdesign}		% Mathefont

% root document
% !TeX root = ../Projektarbeit.tex

% Globale Variablen, insbesondere für die Titelseite
\def\title{Mars Science Laboratory Curiosity Rover}		% Titel der Arbeit
\def\titleFrontpage{Research Lab for Deep Learning \\ Mars Science Laboratory Curiosity Rover}
\def\doctype{Studienarbeit (T2\_000)}		% Art des Dokumentes, z.B. Studienarbeit, Bachelorarbeit, Bericht, ...
\def\studiengang{Angewandte Informatik}		% Studiengang
\def\author{Niklas Koopmann}				% Autor des Dokumentes
\def\matrikelNr{9742503}					% Autor des Dokumentes
\def\kursKurz{MOS-TINF17B}					% Autor des Dokumentes
\def\bearbeitungsZeit{24 Wochen}			% Bearbeitsungszeitraum
\def\ausbildungsfirma{Deutsche Bundesbank} 	% Ausbildungsfirma
\def\betreuerFirma{}						% Betreuer der Ausbildungsfirma
\def\betreuerDH{Dr. Carsten Müller}			% Betreuer der Dualen Hochschule
\def\abgabeDatum{\today}					% Abgabedatum der Arbeit

% % % Abstände bei Kapitelüberschriften ändern
\renewcommand*{\chapterheadstartvskip}{\vspace*{-5mm}}   % % % Freiraum vor der Überschrift
%\renewcommand*{\chapterheadendvskip}{\vspace{1cm}}      % % % Freiraum nach der Überschrift

% moved: Links im Dokument (to bottom)

\usepackage{scrhack}		 % % % Workaround für \float@addtolists Fehler

% % % Für das Literaturverzeichnis
\usepackage[
%style=authoryear-icomp,	 % % % Zitierstil - nur 1 verwenden!
style=numeric,				 % % % Zitierstil - nur 1 verwenden!
%style=authoryear,			 % % % Zitierstil - nur 1 verwenden!
isbn=true,                   % % % ISBN nicht anzeigen oder Verbergen, gleiches geht mit nahezu allen anderen Feldern
pagetracker=true,            % % % ebd. bei wiederholten Angaben (false=ausgeschaltet, page=Seite, spread=Doppelseite, true=automatisch)
maxbibnames=5,               % % % maximale Namen, die im Literaturverzeichnis angezeigt werden
maxcitenames=3,              % % % maximale Namen, die im Text angezeigt werden, ab 4 wird "u.a." nach den ersten Autor angezeigt
autocite=inline,             % % % regelt Aussehen für \autocite (inline=\parencite)
block=space,                 % % % kleiner horizontaler Platz zwischen den Feldern
backref=true,                % % % Seiten anzeigen, auf denen die Referenz vorkommt
backrefstyle=three+,         % % % fasst Seiten zusammen, z.B. S. 2f, 6ff, 7-10
date=short,                  % % % Datumsformat
backend=bibtex,	             % % % Beunutze das biber-Backend zur Erstellung an Stelle von bibtex
firstinits=true
]{biblatex}
\DefineBibliographyStrings{ngerman}{andothers={et\addabbrvspace al\adddot}}      % et al. statt u.a.
\setlength{\bibitemsep}{1em}	% % % Abstand zwischen den Literaturangaben
\setlength{\bibhang}{2em}		% % % Einzug nach jeweils erster Zeile
\addbibresource{literatur.bib}	% % % Literaturdatenbank einlesen
\bibliography{literatur}        % % % Literaturdatenbank einlesen

\usepackage{xcolor}
\definecolor{monospacebackgroundcolor}{RGB}{230,230,240}
\definecolor{monospacecolor}{RGB}{50,0,0}
\definecolor{bbkblau}{RGB}{0,98,161}
\definecolor{bbkgrau}{RGB}{219,219,219}
\definecolor{bbkhellgruen}{RGB}{192,217,110}
\definecolor{bbkdunkelgruen}{RGB}{134,195,135}
	
% % % Für das Abkürzungsverzeichnis:
\usepackage[footnote]{acronym}



% % % Schriften und Layout
\usepackage{lmodern}		% Standardschriftart von Latex
\usepackage[expansion=true, protrusion=true]{microtype}	% Abstände verbessern
\usepackage[babel,german=quotes]{csquotes}



% % % Für die Einbindung von Grafiken
\usepackage{graphicx}	% Für Grafiken allgemein
\usepackage{epstopdf}	% Für *.eps-Grafiken mit pdfLaTeX
%\usepackage{svg}		% Für *.svg-Grafiken
\usepackage{caption}	% Verbesserungen bei der Beschriftung von Bildern und Tabellen
%\usepackage{float}		% Verbesserung für fließende Objekte
%\usepackage[extendedchars, encoding, multidot, space, filenameencoding=utf8]{grffile}
\usepackage{wrapfig}
\usepackage{pdfpages}



% % % Für Tabellen
\usepackage{typearea}	
\usepackage{array}		% Basispaket für Tabellen
\usepackage{ltxtable}	% Erweiterung für Tabellen
\usepackage{booktabs}	% Verbesserung für die Darstellung von Tabellen
\usepackage{enumitem}
\usepackage{makecell}	% \thead-Befehl in Tabellen
\usepackage{multirow}
\usepackage{hhline}


% % % URLs ordentlich darstellen
\usepackage{url}

% % % Blindtextgenerator, wird in Dokumenten normalerweise nich benötigt
\usepackage{blindtext}

% % % Seitenabmessungen und Randabstand
\usepackage[ %
%width=210.00mm, height=297.00mm, %
left=2.80cm, right=2.60cm, top=2.60cm, bottom=2.60cm]{geometry}



% % % Für Mathe:
\usepackage{amsmath}	% Mathematische Konstrukte
\usepackage{amsfonts}	% Mathematische Schriftarten
\usepackage{amssymb}	% Mathematische Symbole



% % % Für Quelltexte
\usepackage{listings}
\usepackage{color}

% % % Bäume
\usepackage{tikz}
\usetikzlibrary{positioning}
\usetikzlibrary{trees}
\usetikzlibrary{matrix}


\definecolor{colordcodebackground}{rgb}{0.9,0.9,0.9}
\definecolor{colorcodegreen}{rgb}{0,0.4,0}
\definecolor{colorcodegrey}{rgb}{0.4,0.4,0.4}
\definecolor{colorcodestring}{rgb}{0.58,0.0,0.82}
\definecolor{colorcodekeys}{rgb}{0.0,0.0,0.8}
\definecolor{colorcoderule}{rgb}{0.65,0.65,0.65}

\lstset{ %
	backgroundcolor=\color{colordcodebackground},   % choose the background color; you must add \usepackage{color} or \usepackage{xcolor}
	basicstyle=\ttfamily\footnotesize,  % the size of the fonts that are used for the code
	breakatwhitespace=false,         % sets if automatic breaks should only happen at whitespace
	breaklines=true,                 % sets automatic line breaking
	captionpos=b,                    % sets the caption-position to bottom
	commentstyle=\color{colorcodegreen},    % comment style
	deletekeywords={...},            % if you want to delete keywords from the given language
	escapeinside={\%*}{*)},          % if you want to add LaTeX within your code
	extendedchars=true,              % lets you use non-ASCII characters; for 8-bits encodings only, does not work with UTF-8
	frame=single,	                 % adds a frame around the code
	keepspaces=true,                 % keeps spaces in text, useful for keeping indentation of code (possibly needs columns=flexible)
	keywordstyle=\color{colorcodekeys},       % keyword style
	%language=Java,                	 % the language of the code
	otherkeywords={},           % if you want to add more keywords to the set
	numbers=left,                    % where to put the line-numbers; possible values are (none, left, right)
	numbersep=5pt,                   % how far the line-numbers are from the code
	numberstyle=\color{colorcodegrey}, 	 % the style that is used for the line-numbers
	rulecolor=\color{colorcoderule},         % if not set, the frame-color may be changed on line-breaks within not-black text (e.g. comments (green here))
	showspaces=false,                % show spaces everywhere adding particular underscores; it overrides 'showstringspaces'
	showstringspaces=false,          % underline spaces within strings only
	showtabs=false,                  % show tabs within strings adding particular underscores
	stepnumber=1,                    % the step between two line-numbers. If it's 1, each line will be numbered
	stringstyle=\color{colorcodestring},     % string literal style
	tabsize=4,	                   	 % sets default tabsize to 2 spaces
	title=\lstname,                   % show the filename of files included with \lstinputlisting; also try caption instead of title
	float
} % max 73 Zeichen je Zeile
\renewcommand{\lstlistlistingname}{Quelltextverzeichnis}
\renewcommand{\lstlistingname}{Quelltext}
% http://tex.stackexchange.com/questions/89574/language-option-supported-in-listings
\lstdefinelanguage{js}{
	keywords={break, case, catch, continue, debugger, default, delete, do, else, false, finally, for, function, if, in, instanceof, new, null, return, switch, this, throw, true, try, typeof, var, void, while, with},
	morecomment=[l]{//},
	morecomment=[s]{/*}{*/},
	morestring=[b]',
	morestring=[b]",
	ndkeywords={class, export, boolean, throw, implements, import, this},
	keywordstyle=\color{blue}\bfseries,
	ndkeywordstyle=\color{darkgray}\bfseries,
	identifierstyle=\color{black},
	commentstyle=\color[rgb]{0,0.4,0}\ttfamily,
	stringstyle=\color[rgb]{0.7,0.0,0.8}\ttfamily,
	sensitive=true
}


% % % Kopf- und Fußzeile
\usepackage{scrpage2}
\pagestyle{scrheadings}
\clearscrheadings
%\ifoot{\leftmark}{\leftmark}
\ifoot{}{}
%\ihead{\rightmark}{\rightmark} % % % Rightmark oben links
\ohead{\rightmark}{\rightmark} % % % Rightmark oben links
%\ohead[\pagemark]{\pagemark}  % % % Seitenzahl oben rechts
\ofoot[\pagemark]{\pagemark}   % % % Seitenzahl unten rechts
\cfoot[]{}					   % % % Leere Mitte unten
\automark[section]{chapter}	   % % % [Leftmark] und {Rightmark} setzen







% % % Layout-Optionen:
\renewcommand*{\headfont}{\normalfont}
%\renewcommand*{\multicitedelim}{\addsemicolon\space}
%\renewcommand*{\headrulewidth}{0pt}
\renewcommand*{\arraystretch}{1.5}



% % % Globaler Zeilenabstand: 1,5-zeilig
\usepackage[onehalfspacing]{setspace}
%\setlength{\parskip}{1.5em}
\recalctypearea



% % % Makrobefehle

% Verweis 
% Befehl \see{label}
% #1 : reflabel
\def\see#1{ %
	\text{($\rightarrow$}\space\nameref{#1}\space\text{auf S.}\space\pageref{#1}\text{)}
}

% Bild
% Befehl \bild{Bilddatei}{Beschriftung}{label}
% #1 : Bilddatei
% #2 : Beschriftung / Bildunterschrift
% #3 : Label
\def\bild#1#2#3{ %
	\begin{figure}[h]
		\centering
		\includegraphics[width=0.9\textwidth]{#1}
%		\vspace{5mm}
		\caption{#2}
		\label{#3}
	\end{figure}
}

% Code
% Befehl \code{Quelldatei}{Beschriftung}{label}{Sprache}
\def\code#1#2#3#4{ %
	\begin{center}
	\centering
	\lstset{language=#4,caption={#2},label=#3}
	\singlespacing
	\lstinputlisting[language=#4]{#1}
	\end{center}
}

% Monospace
% Befehl \mono{Text}
\newcommand{\mono}[1]{
	\texttt{\colorbox{monospacebackgroundcolor}{\textcolor{monospacecolor}{#1}}}
	%\texttt{\textcolor{monospacecolor}{#1}}
	%\texttt{#1}
}

\newcommand{\signature}[1]{%
	\parbox{\textwidth}{
		%\centering #3 \today\\
		%\vspace{2cm}
		
		\parbox{7cm}{
			\centering
			\rule{6cm}{1pt}\\
			Ort, Datum 
		}
		\hfill
		\parbox{7cm}{
			\centering
			\rule{6cm}{1pt}\\
			#1
		}
	}
} 

% wechselnde Footnotes (Anmerkungen als Buchstaben)
% usage \ftnote{Basistext}{Fußnotentext}{Index der Fußnote}
\newcommand{\ftnote}[3]{
	\renewcommand{\thefootnote}{\alph{footnote}}
	#1\footnote[#3]{#2}
	\renewcommand{\thefootnote}{\arabic{footnote}}
}

%% better: (general command to vertically center horizontal material)
\newcommand*{\vcenteredhbox}[1]{\begingroup
	\setbox0=\hbox{#1}\parbox{\wd0}{\box0}\endgroup}


% % % Links im Dokument erzeugen
%\usepackage[colorlinks, allcolors=blue, breaklinks]{hyperref}
\usepackage[colorlinks, allcolors=black, breaklinks]{hyperref}
%\usepackage[colorlinks, linkcolor = black, citecolor = black, filecolor = black, urlcolor = blue]{hyperref} 