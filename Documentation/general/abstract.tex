\chapter{Abstract}
\label{sec:abstractenglish}

This study documents the complete theoretical and practical development of a model of the Curiosity Mars Rover controllable by voice commands as a proof of concept.
The model is digitally designed and physically created from LEGO elements.
A control software for the rover is implemented in the programming language Python and used on a Raspberry Pi in combination with three BrickPis for operation.
The result shows that a true-to-original and aesthetically pleasing reproduction of Curiosity including a functional six-wheel drive, rocker-bogie suspension system and reliable color-based object recognition is possible.
The experience and conclusions from the proof of concept serve as a basis for a wide range of further developments at DHBW Mosbach, especially in the areas of object recognition using artificial intelligence as well as swarm intelligence.

\vspace{1cm}

\begingroup
\renewcommand{\pagebreak}{}
\renewcommand{\clearpage}{}
\chapter*{Zusammenfassung}
\label{sec:abstract}

Diese Studienarbeit dokumentiert die vollständige theoretische und praktische Entwicklung eines über Sprachkommandos steuerbaren Modells des Curiosity-Mars-Rovers als Machbarkeitsstudie.
Das Modell wird digital modelliert und physisch aus LEGO-Elementen erstellt.
Weiterhin wird eine Steuerungssoftware für den in der Programmiersprache Python implementiert und auf einem Raspberry Pi in Verbindung mit drei BrickPis zur Steuerung genutzt.
Das Ergebnis zeigt, dass eine originalgetreue und formschöne Abbildung von Curiosity inklusive funktionstüchtigem sechsrädrigen Antrieb, Rocker-Bogie-Fahrwerk und zuverlässiger farbbasierter Objekterkennung möglich ist.
Die Erfahrungen und Ergebnisse aus dem Proof of Concept dienen als Grundlage für vielfältige weitere Entwicklungen an der DHBW Mosbach, insbesondere in den Bereichen der Objekterkennung mithilfe künstlicher Intelligenz sowie der Schwarmintelligenz.

\endgroup